\documentclass{article}

\usepackage{geometry}
\usepackage{amsmath}
\usepackage{graphicx}
\usepackage{listings}
\usepackage{hyperref}
\usepackage{multicol}
\usepackage{fancyhdr}
\pagestyle{fancy}
\hypersetup{ colorlinks=true, linkcolor=black, filecolor=magenta, urlcolor=cyan}
\geometry{ a4paper, total={170mm,257mm}, top=20mm, right=20mm, bottom=20mm, left=20mm}
\setlength{\parindent}{0pt}
\setlength{\parskip}{1em}
\renewcommand{\headrulewidth}{0pt}
\lhead{ITB - IEEEXtreme 14.0 Selection}
\fancyfoot[CE,CO]{\thepage}
\lstset{
    basicstyle=\ttfamily\small,
    columns=fixed,
    extendedchars=true,
    breaklines=true,
    tabsize=2,
    prebreak=\raisebox{0ex}[0ex][0ex]{\ensuremath{\hookleftarrow}},
    frame=none,
    showtabs=false,
    showspaces=false,
    showstringspaces=false,
    prebreak={},
    keywordstyle=\color[rgb]{0.627,0.126,0.941},
    commentstyle=\color[rgb]{0.133,0.545,0.133},
    stringstyle=\color[rgb]{01,0,0},
    captionpos=t,
    escapeinside={(\%}{\%)}
}

\begin{document}

\begin{center}
    \section*{Z. Menghitung Perkalian Faktor-Faktor}

    \begin{tabular}{ | c c | }
        \hline
        Batas Waktu  & 2s \\    % jangan lupa ganti time limit
        Batas Memori & 256MB \\  % jangan lupa ganti memory limit
        \hline
    \end{tabular}
\end{center}

\subsection*{Deskripsi}

Contoh memformat teks: \textbf{bold}, \textit{italic}, \underline{underline}, $x$.

Contoh membuat persamaan:

\[ x_{n+1} = x_{n} - \frac{f(x_{n})}{f'(x_{n})} \]

\begin{enumerate}
    \setlength\itemsep{0pt}
    \item Contoh penomoran.
    \item Contoh penomoran.
\end{enumerate}

\begin{itemize}
    \setlength\itemsep{0pt}
    \item Contoh membuat poin-poin.
    \item Contoh membuat poin-poin.
\end{itemize}

\begin{center}
    Teks rata tengah
    % Contoh gambar:
    % \includegraphics[width=300px]{image-1}
\end{center}

\subsection*{Format Masukan}

Baris pertama terdiri dari satu bilangan bulat positif $N$ ($1 \leq N \leq 200.000$), menyatakan banyaknya faktor prima dari bilangan $X$.
Baris berikutnya berisi $N$ buah bilangan $p_1$, $p_2$, $\dots$, $p_N$ ($1 \leq p_i \leq 200.000$) yang merupakan faktor-faktor prima dari $X$. Dijamin tiap $p_i$ merupakan bilangan prima.
$N$ baris berikutnya terdiri dari 3 bilangan, dengan baris ke-$i$ menyatakan bilangan $A_i$, $B_i$, dan $C_i$.

\subsection*{Format Keluaran}

Tuliskan sebuah bilangan yang merupakan bilangan yang diinginkan dalam modulo $10^9 + 7$.
\\

\begin{multicols}{2}
\subsection*{Contoh Masukan}
\begin{lstlisting}
3
5 2 2
\end{lstlisting}
\columnbreak
\subsection*{Contoh Keluaran}
\begin{lstlisting}
-6
\end{lstlisting}
\vfill
\null
\end{multicols}

% \subsection*{Penjelasan}
% Jika dibutuhkan, tambahkan penjelasan di sini

\pagebreak

\end{document}