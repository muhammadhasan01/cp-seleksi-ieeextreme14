\documentclass{article}

\usepackage{geometry}
\usepackage{amsmath}
\usepackage{graphicx}
\usepackage{listings}
\usepackage{hyperref}
\usepackage{multicol}
\usepackage{fancyhdr}
\pagestyle{fancy}
\hypersetup{ colorlinks=true, linkcolor=black, filecolor=magenta, urlcolor=cyan}
\geometry{ a4paper, total={170mm,257mm}, top=20mm, right=20mm, bottom=20mm, left=20mm}
\setlength{\parindent}{0pt}
\setlength{\parskip}{1em}
\renewcommand{\headrulewidth}{0pt}
\lhead{ITB - IEEEXtreme 14.0 Selection}
\fancyfoot[CE,CO]{\thepage}
\lstset{
    basicstyle=\ttfamily\small,
    columns=fixed,
    extendedchars=true,
    breaklines=true,
    tabsize=2,
    prebreak=\raisebox{0ex}[0ex][0ex]{\ensuremath{\hookleftarrow}},
    frame=none,
    showtabs=false,
    showspaces=false,
    showstringspaces=false,
    prebreak={},
    keywordstyle=\color[rgb]{0.627,0.126,0.941},
    commentstyle=\color[rgb]{0.133,0.545,0.133},
    stringstyle=\color[rgb]{01,0,0},
    captionpos=t,
    escapeinside={(\%}{\%)}
}

\begin{document}

\begin{center}
    \section*{Jumlahan Ganjil}

    \begin{tabular}{ | c c | }
        \hline
        Batas Waktu  & 1s \\    % jangan lupa ganti time limit
        Batas Memori & 256MB \\  % jangan lupa ganti memory limit
        \hline
    \end{tabular}
\end{center}

\subsection*{Deskripsi}

Pak Wengki memiliki $N$ buah bilangan bulat $a_1, a_2, ..., a_N$. Pak Wengki kali ini menyuruh Anda untuk menentukan \textit{subsequence} dari $N$ bilangan bulat tersebut yang memiliki jumlah ganjil dan bernilai paling besar. 

Sekedar informasi, \textit{subsequence} adalah sebuah \textit{sequence} yang bisa didapatkan dari \textit{sequence} lain dengan menghapus beberapa elemen tanpa mengganti urutan elemen yang tersisa.
Contoh: [1, 3] merupakan \textit{subsequence} dari [1, 4, 3, 2], tetapi [2, 3] bukan merupakan \textit{subsequence} dari [1, 4, 3, 2].

Tentukanlah jumlah tersebut!

\subsection*{Format Masukan}

Baris pertama terdiri satu buah bilangan bulat $N$ ($1 \leq N \leq 10^5$) yang menyatakan banyaknya bilangan bulat yang dimiliki oleh Pak Wengki.

Baris berikutnya berisi $N$ buah bilangan bulat ($-10^4 \leq a_1, a_2, ..., a_n \leq 10^4$), yaitu bilangan-bilangan yang dimiliki oleh Pak Wengki.

Dipastikan akan ada minimal 1 buah \textit{subsequence} dengan jumlahan ganjil.

\subsection*{Format Keluaran}

Sebuah bilangan yang merupakan jumlahan bilangan-bilangan dari suatu \textit{subsequence} $N$ bilangan yang dimiliki Pak Wengki, bernilai ganjil, dan paling besar.


\\
\begin{multicols}{2}
\subsection*{Contoh Masukan}
\begin{lstlisting}
4
-2 2 -3 1
\end{lstlisting}
\columnbreak
\subsection*{Contoh Keluaran}
\begin{lstlisting}
3
\end{lstlisting}
\vfill
\null
\end{multicols}

% \subsection*{Penjelasan}
% Jika dibutuhkan, tambahkan penjelasan di sini

\pagebreak

\end{document}