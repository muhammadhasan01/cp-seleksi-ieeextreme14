\documentclass{article}

\usepackage{geometry}
\usepackage{amsmath}
\usepackage{graphicx}
\usepackage{listings}
\usepackage{hyperref}
\usepackage{multicol}
\usepackage{fancyhdr}
\pagestyle{fancy}
\hypersetup{ colorlinks=true, linkcolor=black, filecolor=magenta, urlcolor=cyan}
\geometry{ a4paper, total={170mm,257mm}, top=20mm, right=20mm, bottom=20mm, left=20mm}
\setlength{\parindent}{0pt}
\setlength{\parskip}{1em}
\renewcommand{\headrulewidth}{0pt}
\lhead{ITB - IEEEXtreme 14.0 Selection}
\fancyfoot[CE,CO]{\thepage}
\lstset{
    basicstyle=\ttfamily\small,
    columns=fixed,
    extendedchars=true,
    breaklines=true,
    tabsize=2,
    prebreak=\raisebox{0ex}[0ex][0ex]{\ensuremath{\hookleftarrow}},
    frame=none,
    showtabs=false,
    showspaces=false,
    showstringspaces=false,
    prebreak={},
    keywordstyle=\color[rgb]{0.627,0.126,0.941},
    commentstyle=\color[rgb]{0.133,0.545,0.133},
    stringstyle=\color[rgb]{01,0,0},
    captionpos=t,
    escapeinside={(\%}{\%)}
}

\begin{document}

\begin{center}
    \section*{A. Tuan Mor dan Dua Buah Kata}

    \begin{tabular}{ | c c | }
        \hline
        Batas Waktu  & 1s \\  
        Batas Memori & 256MB \\
        \hline
    \end{tabular}
\end{center}

\subsection*{Deskripsi}

Tuan Mor diberi dua buah kata $A$ dan $B$. Ia dapat melakukan beberapa operasi pada kedua kata tersebut. Setiap operasi terdiri atas langkah-langkah berikut:

\begin{itemize}
    \setlength\itemsep{0pt}
    \item Pilih salah satu dari kata $A$ atau $B$ yang tidak kosong (jumlah karakter pada kata yang dipilih tidak boleh bernilai nol)
    \item Hapus karakter pertama dari kata yang dipilih tersebut.
\end{itemize}

Sebagai contoh, apabila $A = \text{saya}$ dan $b = \text{anda}$, maka pada operasi pertama Tuan Mor dapat membuat $(A,B)$ menjadi $(\text{aya}, \text{anda})$ atau $(\text{saya}, \text{nda})$.

Tugas Tuan Mor adalah membuat kedua kata $A$ dan $B$ menjadi sama dengan melakukan sejumlah operasi tersebut seminimal mungkin. Kedua kata tersebut dapat menjadi kosong di akhir. Bantulah Tuan Mor untuk menentukan jumlah operasi minimal yang harus dilakukan.

\subsection*{Format Masukan}

Baris pertama dan kedua berisi kata $A$ dan $B$ secara berturut-turut ($1 \leq |A|, |B| \leq 200.000$). Kedua kata dijamin hanya terdiri atas \textit{lowercase alphabet}.  

\subsection*{Format Keluaran}

Sebuah bilangan yang merupakan jumlah operasi minimal yang dibutuhkan.
\\

\begin{multicols}{2}
\subsection*{Contoh Masukan 1}
\begin{lstlisting}
mengapa
kenapa
\end{lstlisting}
\columnbreak
\subsection*{Contoh Keluaran 1}
\begin{lstlisting}
7
\end{lstlisting}
\vfill
\null
\end{multicols}

\begin{multicols}{2}
\subsection*{Contoh Masukan 2}
\begin{lstlisting}
wkwkwkkwk
wwkwkwk
\end{lstlisting}
\columnbreak
\subsection*{Contoh Keluaran 2}
\begin{lstlisting}
10
\end{lstlisting}
\vfill
\null
\end{multicols}

\subsection*{Penjelasan}
Pada contoh masukan 1, setelah tujuh langkah, Tuan Mor dapat mengubah kedua kata tersebut menjadi kata 'napa'.
Dapat dibuktikan bahwa minimal dibutuhkan tujuh buah langkah pada kasus uji ini.

\pagebreak

\end{document}