\documentclass{article}

\usepackage{geometry}
\usepackage{amsmath}
\usepackage{graphicx}
\usepackage{listings}
\usepackage{hyperref}
\usepackage{multicol}
\usepackage{fancyhdr}
\pagestyle{fancy}
\hypersetup{ colorlinks=true, linkcolor=black, filecolor=magenta, urlcolor=cyan}
\geometry{ a4paper, total={170mm,257mm}, top=20mm, right=20mm, bottom=20mm, left=20mm}
\setlength{\parindent}{0pt}
\setlength{\parskip}{1em}
\renewcommand{\headrulewidth}{0pt}
\lhead{ITB - IEEEXtreme 14.0 Selection}
\fancyfoot[CE,CO]{\thepage}
\lstset{
    basicstyle=\ttfamily\small,
    columns=fixed,
    extendedchars=true,
    breaklines=true,
    tabsize=2,
    prebreak=\raisebox{0ex}[0ex][0ex]{\ensuremath{\hookleftarrow}},
    frame=none,
    showtabs=false,
    showspaces=false,
    showstringspaces=false,
    prebreak={},
    keywordstyle=\color[rgb]{0.627,0.126,0.941},
    commentstyle=\color[rgb]{0.133,0.545,0.133},
    stringstyle=\color[rgb]{01,0,0},
    captionpos=t,
    escapeinside={(\%}{\%)}
}

\begin{document}

\begin{center}
    \section*{B. Harta Karun} % ganti judul soal

    \begin{tabular}{ | c c | }
        \hline
        Batas Waktu  & 1s \\    % jangan lupa ganti time limit
        Batas Memori & 256MB \\  % jangan lupa ganti memory limit
        \hline
    \end{tabular}
\end{center}

\subsection*{Deskripsi}

Faddila berhasil menemukan tempat harta karun di gua tersembunyi, sayangnya gua ini memiliki \textit{trap} yang membuat beberapa harta karun hangus tertelan bumi jika tidak cepat diambil. Faddila mempunyai kekuatan untuk mengestimasi nilai $W_i$, yakni waktu yang dibutuhkan untuk mengambil harta karun ke-$i$ dalam satuan detik. Selain itu, Faddila juga dapat mengestimasi nilai $H_i$, yakni waktu dalam detik saat harta karun ke-$i$ akan hangus dan tidak bisa lagi diambil. Beberapa harta karun lebih berharga dari harta karun yang lain, Faddila bisa juga mengestimasi nilai $V_i$, yakni nilai harga suatu harta karun.

Carilah harga total maksimum yang bisa didapatkan Faddila dengan mengambil beberapa harta karun secara berurutan! 

Sebagai contoh, jika Faddila mengambil harta karun ke-$i$ lalu harta karun ke-$j$, maka harta karun ke-$i$ akan diambil dalam waktu $W_i$ detik dan harta karun ke-$j$ akan diambil dalam waktu $W_i + W_j$ detik.

\subsection*{Format Masukan}

Baris pertama berisi satu bilangan bulat positif, $N$ ($1 \leq N \leq 100$) — banyaknya harta karun yang ada di gua.

$N$ baris berikutnya berisi tiga bilangan bulat positif, dengan baris ke-$i$ berisi $W_i, H_i, V_i$, menyatakan waktu yang dibutuhkan untuk mengambil harta karun ke-$i$, waktu saat harta karun ke-$i$ hangus, dan nilai harga harta karun ke-$i$. ($1 \leq W_i \leq 20, 1 \leq H_i \leq 2000, 1 \leq V_i \leq 10^5$).

\subsection*{Format Keluaran}

Keluarkan satu baris berisi satu bilangan bulat positif, menyatakan harga total maksimum harta karun yang dapat diambil.
\\

\begin{multicols}{2}
\subsection*{Contoh Masukan}
\begin{lstlisting}
3
3 7 4
2 6 5
3 4 6
\end{lstlisting}
\columnbreak
\subsection*{Contoh Keluaran}
\begin{lstlisting}
11
\end{lstlisting}
\vfill
\null
\end{multicols}

\subsection*{Penjelasan}
% Jika dibutuhkan, tambahkan penjelasan di sini
Pengambilan optimal yang dapat dilakukan Faddila adalah mengambil harta karun ke-3 dalam waktu 3 detik ($3 \leq H_3$), lalu kemudian mengambil harta karun ke-2 dalam waktu 2 detik ($3 + 2 \leq H_2$). Sehingga harga total maksimum yang didapatkan adalah $V_3 + V_2 = 6 + 5 = 11$.

\pagebreak

\end{document}