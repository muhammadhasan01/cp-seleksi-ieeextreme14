\documentclass{article}

\usepackage{geometry}
\usepackage{amsmath}
\usepackage{graphicx}
\usepackage{listings}
\usepackage{hyperref}
\usepackage{multicol}
\usepackage{fancyhdr}
\pagestyle{fancy}
\hypersetup{ colorlinks=true, linkcolor=black, filecolor=magenta, urlcolor=cyan}
\geometry{ a4paper, total={170mm,257mm}, top=20mm, right=20mm, bottom=20mm, left=20mm}
\setlength{\parindent}{0pt}
\setlength{\parskip}{1em}
\renewcommand{\headrulewidth}{0pt}
\lhead{ITB - IEEEXtreme 14.0 Selection}
\fancyfoot[CE,CO]{\thepage}
\lstset{
    basicstyle=\ttfamily\small,
    columns=fixed,
    extendedchars=true,
    breaklines=true,
    tabsize=2,
    prebreak=\raisebox{0ex}[0ex][0ex]{\ensuremath{\hookleftarrow}},
    frame=none,
    showtabs=false,
    showspaces=false,
    showstringspaces=false,
    prebreak={},
    keywordstyle=\color[rgb]{0.627,0.126,0.941},
    commentstyle=\color[rgb]{0.133,0.545,0.133},
    stringstyle=\color[rgb]{01,0,0},
    captionpos=t,
    escapeinside={(\%}{\%)}
}

\begin{document}

\begin{center}
    \section*{G. Tuan Mor Menebak Bilangan}

    \begin{tabular}{ | c c | }
        \hline
        Batas Waktu  & 2s \\ 
        Batas Memori & 256MB \\ 
        \hline
    \end{tabular}
\end{center}

\subsection*{Deskripsi}

Tuan Mor sedang bermain tebak bilangan. Pada setiap ronde ia diberikan sebuah bilangan bulat positif $X$, dan akan menebak suatu bilangan yang merupakan nilai dari $f(X)$ untuk suatu fungsi $f$. Ia tidak diberitahu definisi dari fungsi $f$ tersebut. Karena nilai $X$ bisa bernilai sangat besar, pada tiap ronde ia hanya diberikan faktor-faktor prima dari $X$. 

Setelah beberapa ronde, ia akhirnya dapat menemukan pola dari fungsi $f$ tersebut, dan menebak bahwa $f(X)$ merupakan perkalian dari seluruh faktor positif dari $X$. Untuk mengecek kebenaran dari tebakannya ini, ia memutuskan untuk menggunakan fungsi ini sebagai fungsi $f$ untuk menebak angka pada ronde berikutnya. Ia meminta anda ikut menghitung untuk meng-\textit{cross-check} jawabannya. Bantulah Tuan Mor untuk menghitung nilai dari bilangan yang akan menjadi tebakannya pada ronde berikutnya.

\subsection*{Format Masukan}

Baris pertama terdiri dari satu bilangan bulat positif $N$ ($1 \leq N \leq 200.000$) yang menyatakan banyaknya faktor prima dari bilangan yang diberikan pada ronde berikutnya.
Baris berikutnya berisi $N$ buah bilangan $p_1$, $p_2$, $\dots$, $p_N$ ($1 \leq p_i \leq 200.000$) yang merupakan faktor-faktor prima dari bilangan yang diberikan tersebut. Dijamin tiap $p_i$ merupakan bilangan prima.

\subsection*{Format Keluaran}

Tuliskan sebuah bilangan yang merupakan bilangan yang diinginkan dalam modulo $10^9 + 7$.
\\

\begin{multicols}{2}
\subsection*{Contoh Masukan 1}
\begin{lstlisting}
2
2 3
\end{lstlisting}
\columnbreak
\subsection*{Contoh Keluaran 1}
\begin{lstlisting}
6
\end{lstlisting}
\vfill
\null
\end{multicols}

\begin{multicols}{2}
\subsection*{Contoh Masukan 2}
\begin{lstlisting}
1
31
\end{lstlisting}
\columnbreak
\subsection*{Contoh Keluaran 2}
\begin{lstlisting}
31
\end{lstlisting}
\vfill
\null
\end{multicols}

\subsection*{Penjelasan}
Pada contoh masukan 1, seluruh faktor positif dari $X = 2 \times 3 = 6$ adalah $\{ 1,2,3,6 \}$. Perkalian mereka semua adalah $36$.

\pagebreak

\end{document}