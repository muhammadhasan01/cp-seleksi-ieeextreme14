\documentclass{article}

\usepackage{geometry}
\usepackage{amsmath}
\usepackage{graphicx}
\usepackage{listings}
\usepackage{hyperref}
\usepackage{multicol}
\usepackage{fancyhdr}
\pagestyle{fancy}
\hypersetup{ colorlinks=true, linkcolor=black, filecolor=magenta, urlcolor=cyan}
\geometry{ a4paper, total={170mm,257mm}, top=20mm, right=20mm, bottom=20mm, left=20mm}
\setlength{\parindent}{0pt}
\setlength{\parskip}{1em}
\renewcommand{\headrulewidth}{0pt}
\lhead{ITB - IEEEXtreme 14.0 Selection}
\fancyfoot[CE,CO]{\thepage}
\lstset{
    basicstyle=\ttfamily\small,
    columns=fixed,
    extendedchars=true,
    breaklines=true,
    tabsize=2,
    prebreak=\raisebox{0ex}[0ex][0ex]{\ensuremath{\hookleftarrow}},
    frame=none,
    showtabs=false,
    showspaces=false,
    showstringspaces=false,
    prebreak={},
    keywordstyle=\color[rgb]{0.627,0.126,0.941},
    commentstyle=\color[rgb]{0.133,0.545,0.133},
    stringstyle=\color[rgb]{01,0,0},
    captionpos=t,
    escapeinside={(\%}{\%)}
}

\begin{document}

\begin{center}
    \section*{H. Menyusun Prajurit} % ganti judul soal

    \begin{tabular}{ | c c | }
        \hline
        Batas Waktu  & 2s \\    % jangan lupa ganti time limit
        Batas Memori & 512MB \\  % jangan lupa ganti memory limit
        \hline
    \end{tabular}
\end{center}

\subsection*{Deskripsi}

Di kerajaan Ganesha, terdapat $N$ prajurit. Setiap prajurit memiliki besar skill \textit{Attack} dan besar skill \textit{Defense} masing-masing. Baru-baru ini, terjadi peperangan besar diantara kerajaan, oleh karena itu kerajaan Ganesha perlu membuat suatu susunan prajurit. Raja ingin terdapat tepat $X$ prajurit yang berfungsi untuk menyerang dan tepat $Y$ prajurit yang berfungsi untuk bertahan di kerajaan.

Sudah menjadi peraturan umum bahwa kekuatan total dari suatu prajurit adalah penjumlahan dari dua nilai, yakni jumlah skill \textit{Attack} dari prajurit yang berfungsi untuk menyerang, dan jumlah skill \textit{Defense} dari prajurit yang berfungsi untuk bertahan.

Raja ingin mengetahui kekuatan total maksimum dari susunan prajurit yang bisa didapatkan. Anda sebagai ahli strategi kerajaan Ganesha, diminta untuk mencarinya!

\subsection*{Format Masukan}

Baris pertama terdiri dari tiga bilangan bulat positif $N, X, Y$ ($1 \leq N \leq 3000, 0 \leq X, Y \leq N, X + Y \leq N$), secara berturut-turut menyatakan banyaknya prajurit yang ada di kerajaan Ganesha, banyaknya prajurit yang perlu berfungsi untuk menyerang, dan banyaknya prajurit yang perlu berfungsi untuk bertahan.

$N$ baris berikutnya berisi dua bilangan bulat positif, dengan baris ke-$i$ berisi bilangan bulat positif $A_i, B_i$ ($1 \leq A_i, B_i \leq 10^5$), yang menyatakan besarnya skill \textit{Attack} dan besarnya skill \textit{Defense} dari prajurit ke-$i$.

\subsection*{Format Keluaran}

Tuliskan satu bilangan bulat positif menyatakan kekuatan total maksimum susunan prajurit yang bisa didapatkan.
\\

\begin{multicols}{2}
\subsection*{Contoh Masukan}
\begin{lstlisting}
5 2 2
1 5
3 3
4 2
5 1
2 4
\end{lstlisting}
\columnbreak
\subsection*{Contoh Keluaran}
\begin{lstlisting}
18
\end{lstlisting}
\vfill
\null
\end{multicols}

\subsection*{Penjelasan}
% Jika dibutuhkan, tambahkan penjelasan di sini

Kekuatan total maksimum yang didapatkan adalah dengan menaruh prajurit 3, dan prajurit 4 untuk menyerang, lalu menaruh prajurit 1 dan prajurit 5 untuk bertahan. Sehingga didapat kekuatan total maksimumnya adalah:

$$A_3+A_4+B_1+B_5=4+5+5+4=18$$

\pagebreak

\end{document}