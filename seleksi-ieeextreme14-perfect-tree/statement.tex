\documentclass{article}

\usepackage{geometry}
\usepackage{amsmath}
\usepackage{graphicx}
\usepackage{listings}
\usepackage{hyperref}
\usepackage{multicol}
\usepackage{fancyhdr}
\pagestyle{fancy}
\hypersetup{ colorlinks=true, linkcolor=black, filecolor=magenta, urlcolor=cyan}
\geometry{ a4paper, total={170mm,257mm}, top=20mm, right=20mm, bottom=20mm, left=20mm}
\setlength{\parindent}{0pt}
\setlength{\parskip}{1em}
\renewcommand{\headrulewidth}{0pt}
\lhead{ITB - IEEEXtreme 14.0 Selection}
\fancyfoot[CE,CO]{\thepage}
\lstset{
    basicstyle=\ttfamily\small,
    columns=fixed,
    extendedchars=true,
    breaklines=true,
    tabsize=2,
    prebreak=\raisebox{0ex}[0ex][0ex]{\ensuremath{\hookleftarrow}},
    frame=none,
    showtabs=false,
    showspaces=false,
    showstringspaces=false,
    prebreak={},
    keywordstyle=\color[rgb]{0.627,0.126,0.941},
    commentstyle=\color[rgb]{0.133,0.545,0.133},
    stringstyle=\color[rgb]{01,0,0},
    captionpos=t,
    escapeinside={(\%}{\%)}
}

\begin{document}

\begin{center}
    \section*{K - Perfect Tree} % ganti judul soal

    \begin{tabular}{ | c c | }
        \hline
        Batas Waktu  & 1s \\    % jangan lupa ganti time limit
        Batas Memori & 256MB \\  % jangan lupa ganti memory limit
        \hline
    \end{tabular}
\end{center}

\subsection*{Deskripsi}

Rooted tree adalah struktur data tree yang memiliki akar. Semua node pada rooted tree memiliki sebuah parent kecuali pada node yang menjadi akarnya.

Pada soal ini, diberikan sebuah rooted tree yang memiliki node 1 sebagai akar. Rooted tree ini memiliki $N$ buah node bernomor $1$ sampai $N$ dimana untuk setiap node $2$ sampai $N$ akan memiliki sebuah node parent $P_i$. 

Selain itu, setiap node juga memiliki nilai $V_i$.

Sebuah perfect tree didefinisikan sebagai tree yang jarak pada setiap pasang nodenya sama dengan selisih nilai yang ada pada mereka.

Untuk setiap subtree, carilah berapa banyak nilai node minimum yang harus diubah agar subtree tersebut termasuk kedalam perfect tree atau nyatakan jika tidak mungkin.


\subsection*{Format Masukan}

Baris pertama berisi bilangan bulat $Q$ ($1 \leq N \leq 10$) yaitu banyaknya kasus uji.

Kemudian untuk setiap kasus uji:

\begin{itemize}
    \item Baris pertama terdiri dari satu bilangan bulat $N$ ($1 \leq N \leq 10^5$) yang menyatakan banyak node pada tree.
    \item Baris kedua terdiri dari N bilangan bulat $V_i$ ($1 \leq V_1 \leq 10^5$) yang menyatakan nilai pada node-$i$.
    \item Baris ketiga sampai $N + 1$ berisi satu bilangan bulat $P_i$ ($1 \leq P_i \leq N$) yang menyatakan parent dari node-$i$ ($2 \leq i \leq N$).
\end{itemize}

\subsection*{Format Keluaran}

Untuk setiap kasus uji:

Pada baris pertama keluarkan "Case #i:" yang menandakan jawaban untuk kasus uji ke-$i$. 

Kemudian, keluarkan $N$ buah baris yang berisi jawaban untuk masing-masing subtree node-$i$. Baris ke-$i$ merupakan jawaban untuk subtree node-$i$ yang berisi satu bilangan bulat $X_i$, yaitu banyak node minimum yang harus diubah nilainya agar subtree menjadi perfect tree. Jika tidak mungkin menjadi perfect tree maka keluarkan -1.
\\

\begin{multicols}{2}
\subsection*{Contoh Masukan}
\begin{lstlisting}
2
3
3 3 2
1
1
3
3 1 50
1
1

\end{lstlisting}
\columnbreak
\subsection*{Contoh Keluaran}
\begin{lstlisting}
Case #1:
1
0
0
Case #2:
2
0
0
\end{lstlisting}
\vfill
\null
\end{multicols}

% \subsection*{Penjelasan}
% Jika dibutuhkan, tambahkan penjelasan di sini

\pagebreak

\end{document}