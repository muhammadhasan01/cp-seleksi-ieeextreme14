\documentclass{article}

\usepackage{geometry}
\usepackage{amsmath}
\usepackage{graphicx}
\usepackage{listings}
\usepackage{hyperref}
\usepackage{multicol}
\usepackage{fancyhdr}
\pagestyle{fancy}
\hypersetup{ colorlinks=true, linkcolor=black, filecolor=magenta, urlcolor=cyan}
\geometry{ a4paper, total={170mm,257mm}, top=20mm, right=20mm, bottom=20mm, left=20mm}
\setlength{\parindent}{0pt}
\setlength{\parskip}{1em}
\renewcommand{\headrulewidth}{0pt}
\lhead{ITB - IEEEXtreme 14.0 Selection}
\fancyfoot[CE,CO]{\thepage}
\lstset{
    basicstyle=\ttfamily\small,
    columns=fixed,
    extendedchars=true,
    breaklines=true,
    tabsize=2,
    prebreak=\raisebox{0ex}[0ex][0ex]{\ensuremath{\hookleftarrow}},
    frame=none,
    showtabs=false,
    showspaces=false,
    showstringspaces=false,
    prebreak={},
    keywordstyle=\color[rgb]{0.627,0.126,0.941},
    commentstyle=\color[rgb]{0.133,0.545,0.133},
    stringstyle=\color[rgb]{01,0,0},
    captionpos=t,
    escapeinside={(\%}{\%)}
}

\begin{document}

\begin{center}
    \section*{Membuat Segitiga}

    \begin{tabular}{ | c c | }
        \hline
        Batas Waktu  & 1s \\    % jangan lupa ganti time limit
        Batas Memori & 256MB \\  % jangan lupa ganti memory limit
        \hline
    \end{tabular}
\end{center}

\subsection*{Deskripsi}

Fang sedang berjalan - jalan di hutan. Dia melihat ada banyak kayu yang jatuh dengan panjang yang beragam. Ada $n$ buah kayu dengan panjang masing-masing $a[1], a[2], a[3], \dots, a[n]$. Berapa banyak cara dia bisa memilih 3 buah kayu dan membuat segitiga dengan kayu - kayu tersebut? Dua buah cara dianggap berbeda jika terdapat setidaknya satu kayu yang memiliki indeks berbeda diantara kedua cara tersebut. Untuk lebih jelasnya, lihat bagian penjelasan.

\subsection*{Format Masukan}

Baris pertama terdiri dari satu bilangan bulat positif $n$ ($1 \leq n \leq 500$), menyatakan banyaknya kayu.
Baris berikutnya berisi $n$ buah bilangan $a[1], a[2], a[3], \dots, a[n]$ ($1 \leq a[i] \leq 2.10^9$ untuk setiap $i$), menyatakan panjang kayu.

\subsection*{Format Keluaran}

Keluarkan 1 buah baris yang berisi banyaknya segitiga yang bisa dibuat.
\\

\begin{multicols}{2}
\subsection*{Contoh Masukan 1}
\begin{lstlisting}
5
1 3 5 7 9
\end{lstlisting}
\columnbreak
\subsection*{Contoh Keluaran 1}
\begin{lstlisting}
3
\end{lstlisting}
\vfill
\null
\end{multicols}

\begin{multicols}{2}
\subsection*{Contoh Masukan 2}
\begin{lstlisting}
4
2 2 2 2
\end{lstlisting}
\columnbreak
\subsection*{Contoh Keluaran 2}
\begin{lstlisting}
4
\end{lstlisting}
\vfill
\null
\end{multicols}

\subsection*{Penjelasan}
Untuk kasus pertama, Fang dapat memilih kayu dengan panjang (3,5,7), (3,7,9), dan (5,7,9).

Untuk kasus kedua, semua triplet kayu dapat dipilih.

\pagebreak

\end{document}