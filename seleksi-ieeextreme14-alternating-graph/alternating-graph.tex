\documentclass{article}

\usepackage{geometry}
\usepackage{amsmath}
\usepackage{graphicx}
\usepackage{listings}
\usepackage{hyperref}
\usepackage{multicol}
\usepackage{fancyhdr}
\pagestyle{fancy}
\hypersetup{ colorlinks=true, linkcolor=black, filecolor=magenta, urlcolor=cyan}
\geometry{ a4paper, total={170mm,257mm}, top=20mm, right=20mm, bottom=20mm, left=20mm}
\setlength{\parindent}{0pt}
\setlength{\parskip}{1em}
\renewcommand{\headrulewidth}{0pt}
\lhead{ITB - IEEEXtreme 14.0 Selection}
\fancyfoot[CE,CO]{\thepage}
\lstset{
    basicstyle=\ttfamily\small,
    columns=fixed,
    extendedchars=true,
    breaklines=true,
    tabsize=2,
    prebreak=\raisebox{0ex}[0ex][0ex]{\ensuremath{\hookleftarrow}},
    frame=none,
    showtabs=false,
    showspaces=false,
    showstringspaces=false,
    prebreak={},
    keywordstyle=\color[rgb]{0.627,0.126,0.941},
    commentstyle=\color[rgb]{0.133,0.545,0.133},
    stringstyle=\color[rgb]{01,0,0},
    captionpos=t,
    escapeinside={(\%}{\%)}
}

\begin{document}

\begin{center}
    \section*{J. Alternating Pot}

    \begin{tabular}{ | c c | }
        \hline
        Batas Waktu  & 1s \\  
        Batas Memori & 256MB \\
        \hline
    \end{tabular}
\end{center}

\subsection*{Deskripsi}

Tuan Fan gemar mengoleksi tanaman.
Di rumahnya, ia memiliki $N$ buah pot yang tersusun sejajar, dimana terdapat pot yang terisi tanaman maupun yang masih kosong.
Karena ia sedang bosan, maka ia ingin membuat susunan pot tersebut menjadi sebuah susunan yang indah.
Didefinisikan susunan pot yang indah adalah susunan pot dimana:

\begin{itemize}
    \setlength\itemsep{0pt}
    \item Banyak pot di antara 2 buah pembatas adalah maksimal $K$.
    \item Pot di antara 2 buah pembatas bukan merupakan \textit{alternating pot}
\end{itemize}

Pot-pot di antara 2 buah pembatas didefinisikan sebagai \textit{alternating pot} jika untuk setiap $i$ berlaku salah satu dari $i$ dan $i+1$ memiliki tanaman namun tidak keduanya.

Jika di antara 2 buah pembatas hanya terdapat 1 buah pot, maka ia tidak dihitung sebagai \textit{alternating pot}.

Pada awalnya, hanya terdapat 2 buah pembatas, yaitu pada ujung-ujung dari pot tersebut.
Tuan Fan dapat menambahkan pembatas di antara dua buah pot.
Bantu Tuan Fan mencari tahu berapa minimal pembatas yang harus ditambahkan!

\subsection*{Format Masukan}

Baris pertama berisi $T$ ($1 \leq T \leq 200.000$) yang menandakan banyaknya testcase.
Untuk setiap testcase, terdapat 2 baris, dimana baris pertama adalah $N$ dan $K$ ($1 \leq K \leq N \leq 1.000$) dengan $N$ adalah banyak dari pot dan K adalah banyak pot maksimal yang dapat diapit oleh 2 pembatas.
Baris kedua adalah sebuah string dengan panjang $N$ yang menandakan pot-pot yang ada, dimana $1$ menandakan bahwa pot tersebut terisi tanaman dan $0$ menandakan bahwa pot tersebut kosong.

\subsection*{Format Keluaran}

Untuk setiap testcase, keluarkan minimal pembatas yang dibutuhkan agar susunan pot yang ada menjadi indah.
\\

\begin{multicols}{2}
\subsection*{Contoh Masukan}
\begin{lstlisting}
2
5 2
11010
3 3
110
\end{lstlisting}
\columnbreak
\subsection*{Contoh Keluaran}
\begin{lstlisting}
3
0
\end{lstlisting}
\vfill
\null
\end{multicols}

\subsection*{Penjelasan}
Untuk testcase pertama, Tuan Fan harus menaruh pembatas pada seluruh tempat selain pada jeda antara pot ke-1 dan ke-2.
Untuk testcase kedua, Tuan Fan tidak perlu menaruh pembatas karena susunan pot tersebut sudah merupakan susunan pot yang indah.

\pagebreak

\end{document}